\documentclass[10pt]{article}
%\usepackage[cmex10]{amsmath}
\usepackage{array}
\usepackage{mdwmath}
\usepackage{wrapfig}
%\usepackage{ae}
\usepackage[T1]{fontenc}
\usepackage{tgpagella}
\usepackage{float}
\usepackage{enumitem}
%\usepackage[latin1]{inputenc}
\usepackage{mdwtab}
\usepackage{amssymb}
\usepackage{eqparbox}
\usepackage{eulervm}
\usepackage{geometry}
\usepackage{subfig}
\usepackage{graphicx}
\usepackage{setspace}
\usepackage{url}
%\usepackage{lmodern}
\geometry{lmargin=2cm,tmargin=1cm,rmargin=1cm,bmargin=2cm}
%opening
\setlength\parindent{0pt}

\begin{document}
\pagestyle{empty}
\begin{titlepage}
     {\Large{\textbf{Jos\'e Vin\'icius de Miranda Cardoso}}}
     \vspace{.5cm}

    \begin{minipage}[b]{8cm}
     Undergraduate Student\\
     Federal University of Campina Grande, Brazil\\
     Department of Electrical Engineering\\
     Campina Grande, Brazil
    \end{minipage}
    \hspace{4cm}
    \begin{minipage}[b]{4cm}
        \texttt{jvmirca@gmail.com}\\ \texttt{http://mirca.github.io}
    \end{minipage}


\section*{Education}

\begin{description}
 \item[2011] \emph{Undergraduate in progress in Electrical Engineering} \\\textbf{Federal University of Campina Grande}, Brazil\\ Advisor: Marcelo Sampaio de Alencar
 \item[Fall 2014 -- Spring 2015] \emph{Visiting Student -- Electrical Engineering and Computer Science} \\\textbf{The Catholic University of America}, USA\\
\textbf{University of Maryland at College Park}, USA \\ Brazil Scientific Mobility
 Program, Fully funded scholarship recipient \\ Advisors: Duilia F. de Mello and Jandro L. Abot
 \item[2007 -- 2010] \emph{Technical Degree in Informatics} \\ \textbf{Federal Institute of Education, Science and Technology of Para\'iba}, Brazil \\ Advisor: Carlos Danilo Miranda Regis
\end{description}

\section*{Professional Experience}
\begin{description}
\item[2017 -- Current] \emph{Scientific Software Engineering Intern}
\\\textbf{NASA Ames Research Center}, Silicon Valley, USA
\\Kepler/K2 Guest Observer Office
\\Mentor: Geert Barentsen
\item[Summer 2016] \emph{Software Developer at Google Summer of Code}
\\\textbf{Google Summer of Code -- The AstroPy Project}
\\Mentors: Erik Tollerud, Hans Moritz G\"unther, and Brigitta Sipocz
\item[Spring 2015] \emph{Undergraduate Teaching Assistant}
\\ \emph{Probability and Statistics for Electrical Engineering and Computer Science}
\\\textbf{Federal University of Campina Grande}, Brazil
\item[Fall 2015 -- 2016] \emph{Undergraduate Research Assistant}
\\\textbf{Institute for Advanced Studies in Communications}, Brazil
\\Mentor: Marcelo Sampaio Alencar
\item[Summer 2015] \emph{Undergraduate Guest Researcher}
\\\textbf{National Institute of Standards and Technology}, Gaithersburg, USA
\\Center for Nanoscale Science and Technology
\\Nanofabrication Research Group
\\Mentor: Marcelo Ishihara Davan\c co
\item[2011 -- 2014] \emph{Undergraduate Research Assistant}
\\\textbf{Institute for Advanced Studies in Communications}, Brazil
\\Mentor: Marcelo Sampaio Alencar
\end{description}

\section*{Projects}

    \begin{minipage}[b]{18cm}
    The Astropy Project/ Google Summer of Code\\
        \emph{May'16 -- Aug'16} Point spread function photometry for fitting overlapping stars simultaneously
    \end{minipage}\\

    \begin{minipage}[b]{18cm}
    National Institute of Science and Technology, USA\\
        \emph{May'15 -- Aug'15} Parameter estimation for photoactivated localization microscopy
    \end{minipage}

    \begin{minipage}[b]{18cm}
    Institute of Advanced Studies in Communications, Brazil\\
        \emph{2016 -- 2016} Statistical characterization of free space optical channels\\
        \emph{2015 -- 2016} Signal detection in generalized fading channels\\
        \emph{2013 -- 2014} Multiplatform software for objective stereoscopic image and video quality assessment\\
        \emph{2012 -- 2013} Stereoscopic video quality estimation using objective algorithms\\
        \emph{2011 -- 2012} Development of a novel objective algorithm for video quality assessment\\
    \end{minipage}

\section*{Publications}
Refer to \url{https://mirca.github.io/publications}

\section*{Competencies}
\begin{description}
    \item[Software:] Python (numpy, scipy, pandas, scikit-learn), git/GitHub, C/C++, Unix shell
    \item[Favourite courses:] Stochastic Processes, Information Theory, Random Signal Theory, Estimation and Detection Theory
    \item[Languages:] Native Portuguese, Fluent English
\end{description}

\section*{Awards}
\begin{enumerate}
  \item Selected to the Python in Astronomy Conference, Leiden, The Netherlands, 2017
  \item Selected to the S\~ao Paulo School of Advanced Science on Nanophotonics, S\~ao Paulo, Brazil, 2016
  \item Travel Grant Recipient, IEEE Antennas and Propagation Symposium, Puerto Rico, 2016
  \item Young Author Recognition Award, International Telecommunication Union, ITU Kaleidoscope 2015
  \item Young Author Recognition Award, International Telecommunication Union, ITU Kaleidoscope 2014
  \item The paper ``SQUALES: A QT-based Application for Full-Reference Objective Stereoscopic Video Quality Measurement'' was one of the six papers nominated for Best Paper Award at ITU Kaleidoscope 2014
\end{enumerate}

\section*{Additional Information}
\begin{itemize}
    \item[--] Member of the AstroPy software development community.
    \item[--] Participated at the \textit{PSF Photometry and Software Workshop}, Space Telescope Science Institute, Baltimore, 2017.
    \item[--] Participated in the IEEEXtreme 24-Hours Programming Competition in 2013, 2014, 2015, and 2016.
    \item[--] Student of the week on the IEEE Students Facebook webpage.
    \item[--] Attended NASA Ames Machine Learning Workshop, 2017.
\end{itemize}

\end{titlepage}

\end{document}
